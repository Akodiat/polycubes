\begin{savequote}[8cm]
You live and learn. At any rate, you live.
  \qauthor{--- Douglas Adams, Mostly Harmless}
\end{savequote}

\chapter{Conclusion}

In conclusion, nanostructures self-assembly design has been investigated on both an abstract and a more detailed level. The presented projects have resulted in tools and methods for creating, simulating and analysing self-limiting modular structures with minimal complexity, potentially containing building blocks created in different design software.


\section{Abstract self-assembly and design}

The polycube model provides a simple way to self-assemble and design modular shapes. The stochastic assembler presented in Chapter~\ref{ch:polycubes1} provides a method to quickly evaluate input candidates, ensuring that they give the intended output.

The same chapter also showed how large samples of the input space give valuable insight into the frequency of complex and simple shapes. With low complexity polycubes exponentially more common than their high complexity counterparts (during random sampling), we can imagine how simplicity in nature is not necessarily an evolutionary adaption but simply a more probable outcome.

While Chapter~\ref{ch:polycubes1} showed how to map input rules into output shapes, Chapter~\ref{ch:polycubes2} showed how to find the input rules that make a given shape, making it possible to determine the minimal complexity rule for an intended design. This should facilitate the design of modular experimental structures such as those presented in Chapter~\ref{ch:polycubes_intro}.

\section{Nucleic acid design and simulation}

The oxView online tool presented in Chapter~\ref{ch:oxview} has significantly simplified the setup, visualisation, and analysis of oxDNA simulations. The ability to import and combine structures from different tools facilitates the design of modular multi-component structures such as those presented in Chapter~\ref{ch:oxview_intro}.

\section{Future work}

Other shapes than cubes.

Staged assembly

Given a maximum complexity - for example, a limited number of possible colours - what is the largest bounded structure that can deterministically assemble?

Rule out alternative solutions given by non-deterministic assembly via SAT