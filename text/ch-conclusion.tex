\chapter{Discussion}
\label{ch:conclusion}

There are many exciting avenues of further exploration within both polycube design and oxView development. For polycubes, one perhaps obvious comment is that there is nothing special about cubes in particular, so a more general model of self--assembling patchy shapes could prove a useful tool for many problems. The reason I choose cubes, besides them being a natural 3D extension of the polyomino squares, is that they avoid floating--point errors in a cartesian coordinate system. However, that is a minor problem to overcome. With a more general stochastic assembler and with experimental applications such as the ever--larger polyhedral shells presented in Section~\ref{sec:shape-complementary}, we might ask, given a maximum complexity (for example, a limited number of possible colours) what is the largest bounded structure that can deterministically assemble?

Another question for the future is the complexity of polycubes with staged assembly, as investigated by Demaine et al.\ \cite{demaine2008staged}. By assembling a polycube through a hierarchy of separate stages, will an increased complexity in the number of stages decrease the number of colours and species required in an interesting way? The optimisation of other assembly parameters, such as modifying the temperature over time or colours with different interaction strengths, could also be explored for patchy particle assembly. A variable interaction matrix can also be allowed in the SAT solver specification to enable, for example, self-complementary colours.

It may be desirable to incorporate specific building blocks for experiments, perhaps uniquely functionalised by a specific material coating or attachment of a guest molecule, at specific locations. The polycube SAT design method can be extended to allow for the inclusion of such constraints.

%We note that here we focused on constant-temperature assembly. Constant temperatures are appropriate for potential \emph{in vivo} applications, but it is probable that even better assembly yields could be achieved using more complicated temperature protocols \cite{bupathy2022temperature}. We note that previous work has shown that very large multi--component systems, such as single--stranded tiles with targets consisting of hundreds of different species, require controlled cooling from high temperature to ensure rapid assembly and high yields \cite{jacobs2015rational,fonseca2018multi}. The initial high temperature helps the system cross the nucleation barrier; subsequent cooling to low temperature is necessary to combat the high entropic cost of incorporating the last few missing particles into the growing structure. Our stochastic polycube model can be extended to allow for thermally activated building block detachment, thus enabling the investigation of variable-temperature assembly. The design pipeline could also be adapted to investigate the design of different protocols, including staged assembly in multiple pots \cite{tikhomirov2017fractal}, the combination of different interaction strengths \cite{ma2019inverse}, or optimisation of order in which different species are gradually added. 

A further continuation would be to study how the complexity, in terms of the number of species and colours, affect the robustness to defects, such as missing interaction site. Minimal designs are typically more robust to random design errors~\cite{greenbury2016genetic, johnston2021}. In a minimal design, a faulty individual particle is expected to be more easily replaced by another particle of the same species. However, an error affecting an entire species is expected to have larger consequences for species used in multiple locations. 

Finally, the current design pipeline is hindered by the size of the search space. While solutions using few species and colours are readily discovered, larger and more complex shapes requiring large particle libraries need significantly more time and memory to find a solution. Future work could try to mitigate this by separating shapes into smaller sub-shapes, a strategy that could also be linked to staged assembly.

As for the oxView project, one intriguing (but likely time--consuming) possibility would be to integrate other coarse--grained simulation options, for example, mrDNA, similarly to how oxDNA is connected using oxServe. Another ambitious feature would be to include a 2D strand view such as in caDNAno or Adenita, possibly through a connection to the also web--based scadnano.

In conclusion, the design of self--assembling nanostructures has been investigated on both an abstract and a more detailed level. The presented projects have resulted in valuable tools and methods for creating, simulating, and analysing self--limiting modular structures with minimal complexity, potentially containing building blocks created in different design software.