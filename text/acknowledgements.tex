%\subsection*{Personal}

%This is where you thank your advisor, colleagues, and family and friends.

First and foremost, I would like to thank my supervisors, Professor Andrew Turberfield and Professor Ard Louis, for giving me this opportunity to begin with, and for all their help, knowledge, and support. Andrew has provided sound and rational advice whenever I have needed it, helping me to ensure the scientific relevance of my work and making sure I focus on what is important. Likewise, Ard has provided much appreciated advice, inspiration, and ideas for new research directions and collaborations.

I would also like to thank my fellow members of the Turberfield lab, past and present. For their help, for a friendly working environment and for excellent beta testing of my creations; Dr. Jonathan Bath, Rafael Carrascosa Marzo, Dr. Erik Benson, Dr. Seham Helmi, Dr. Antonio Garcia Guerra, Behnam Najafi, Dr. Emma Silvester, Dr. Robert Oppenheimer, Catherine Fan, Alma Chapet--Batlle, Sing Ming Chan, Qian Zhang, and Dr. Rana Abdul Razzak.

I am also very thankful to Professor Petr {\v{S}}ulc, at Arizona State University, for his advice and guidance concerning oxView development, patchy particle simulation, as well as oxRNA. The secondment I spent in the {\v{S}}ulc group was a very valuable and productive time. My fellow oxView developers Erik Poppleton, Dr. Michael Matthies, Jonah Procyk, and Aatmik Mallya also deserve gratitude for their hard and excellent work.

Similarly, I would also like to thank Professor Ebbe Andersen at Aarhus University for my secondment in his group and for introducing me to RNA origami. Also, thank you Néstor Sampedro for all the help and valuable discussions.

Finally, I also appreciate the help and input from Professor Jonathan Doye and his group, including Hannah Fowler, Dr. Domen Prešern and others. The Louis and Doye group meetings have contained many illuminating discussions.

On a more personal note, for her undying support (despite my endless ramblings about polycubes and simulations), my beloved wife Ellen Bohlin simply cannot be thanked enough. Thank you for joining me on yet another adventure and for enduring the times I still had to leave you behind. 

Of course, also want to thank the rest of my family. My father Ulf, who I am certain would also have loved to study here had he been given the opportunity. My mother Okki, for her creativity, her care, and encouragement. My two sisters, Ann-Marie and Ingela, for inspiring me to travel and to study.

I am likewise grateful for the help and encouragement I have received from Ellen's side of the family. My parents-in-law Karl-Johan and Ann-Louise are like an extra set of parents in their support, but I also want to specifically acknowledge Ellen's late grandfather, Anders Bohlin. A botanist in the spirit of Linnaeus, he kept on learning and teaching until the very end.

Our time in Wheatley would not have been the same without such wonderful neighbours; thank you Bruce, Helmer, Harriet and Peter, for making us feel at home and for all the fun we managed to have despite the pandemic.

Furthermore, I also want to thank our taido and karate friends for keeping us sane and in relative shape during these exciting years: Gunnar and (once again) Rafa for Balliol taido, Emily and Asami for welcoming us to the Exeter club, Jesper and Malin for visiting us in Oxford and through zoom. Also, everyone at the Wheatley Ryobu-kai karate club for many fun sessions.

Finally, I have had the privilege of learning from many inspiring supervisors and teachers throughout my studies, so I would also like to thank Professor Claes Andersson, Dr. Alexander Hellervik, Professor Graham Kemp, Olga Balashova, Karin Pleijel, Maria Isaksson, Raido Mäsak, Daniel Lundstedt, and everyone else who have provided me with the knowledge needed to ultimately write this thesis.


%\subsection*{Institutional}

%If you want to separate out your thanks for funding and institutional support, I don't think there's any rule against it.  Of course, you could also just remove the subsections and do one big traditional acknowledgement section.

This project has received funding from the European Union's Horizon 2020 research and innovation programme under the Marie Skłodowska-Curie grant agreement No 765703.
