% Your abstract text goes here. Check your departmental regulations, but generally, this should be less than 300 words. See the beginning of Chapter~\ref{ch:2-litreview} for more.

As nucleic acid nanostructures grow larger and more complex, new tools and methods are needed to facilitate their design. DNA origami structures, for example, are limited by the lengths of their scaffolds but larger assemblies can be bound together by interactions between multiple components. This thesis presents two projects, each approaching the design of such modular structures at a different level of abstraction.

Project 1 presents the \emph{polycube} self--assembly model, where building blocks assemble stochastically using bindings between complementary patches. The assembly of both 2D polyominoes, as well as 3D polycubes, is considered. First, the mapping between input rules (defining the set of available species) and output shapes is investigated, revealing a clear bias toward low--complexity structures. Frequent shapes also tend to be highly modular and symmetric. Secondly, the reversed mapping is explored, presenting a method to find the minimal rule that assembles a specified output shape. Differences in assembly kinetics between possible solutions are investigated using patchy particle simulations, showing that minimal rules can assemble as well as fully addressable rules.

Project 2 presents a more detailed approach to the design of modular structures and individual modules: the \emph{oxView} toolkit for the design, analysis, and visualisation of DNA, RNA and protein nanostructures. While many other design tools exist, oxView makes it easy to import and connect their designs into complete assemblies. Furthermore, oxView allows for free--form editing and rigid--body manipulation. Designs can then be interactively simulated using the oxDNA model, providing a more intuitive understanding of the resulting dynamics.

In conclusion, the design of self--assembling nanostructures has been investigated on both an abstract and a more detailed level. The presented projects have resulted in tools and methods for creating, simulating and analysing modular structures with minimal complexity, potentially containing building blocks created in multiple design software.