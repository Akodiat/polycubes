% Your abstract text goes here.  Check your departmental regulations, but generally this should be less than 300 words.  See the beginning of Chapter~\ref{ch:2-litreview} for more.

As nucleic acid nanostructures grow larger and more complex, new tools and methods are needed to facilitate their design. DNA origami structures, for example, are limited by the lengths of their scaffolds, but can they be joined together into larger multi-component designs. This thesis presents two projects, each approaching the design of such modular structures at a different level of abstraction.

Project 1 presents the \emph{polycube} self-assembly model, where cubic building blocks assemble stochastically using complementary patches. The assembly of both 2D polyominos, as well as 3D polycubes, is exported. First, the mapping between input rules (defining the set of available species) and output shapes is investigated, revealing a clear bias toward low-complexity structures. Secondly, the reversed mapping is explored, presenting a method to find the minimal rule assembling a given output shape.

Project 2 presents a more detailed approach to the design of modular structures and individual modules; the \emph{oxView} toolkit for the design, analysis, and visualisation of DNA, RNA and protein nanostructures. While many other design tools exist, oxView makes it easy to import and connect their designs into complete assemblies. Furthermore, oxView allows for free-form editing and rigid-body manipulation. Designs can then be interactively simulated using the oxDNA model, providing a more intuitive understanding of the dynamics.

In conclusion, highly modular or symmetric shapes and shapes with low complexity tend to be exponentially more frequent within the space of valid input rules. Furthermore, nanostructure modules can be designed using various tools and joined together using oxView, where it is also possible to design structures from scratch. Assemblies can then be simulated and analysed interactively, providing valuable information for further design iterations.