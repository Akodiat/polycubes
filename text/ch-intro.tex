\begin{savequote}[8cm]
Far out in the uncharted backwaters of the unfashionable end of the western spiral arm of the Galaxy lies a small unregarded yellow sun. Orbiting this at a distance of roughly ninety-two million miles is an utterly insignificant little blue green planet whose ape-descended life forms are so amazingly primitive that they still think digital watches are a pretty neat idea.
  \qauthor{--- D. Adams, The Hitchhiker's Guide to the Galaxy}
\end{savequote}

\chapter{Introduction}\label{ch:1-intro}

\minitoc

How do you design and assemble something on the nanoscale? When building something on the macroscale, it is easy to pick the building blocks you want and use your hands to attach them where you want them to be. However, for nanoscale objects, it is difficult to have that level of top-down control. Instead, more success has been had by imitating nature and letting the building blocks assemble themselves. After all, with millions of years of evolution, nature has quite the advantage. This thesis will cover novel methods and tools to design such self-assembling nanostructures.

\section{Thesis structure}
The thesis covers two related projects, both concerning the design and modular self-assembly of nanostructures. Each project is introduced by a separate chapter on its relevant history and background.

The first project, introduced by a background in Chapter~\ref{ch:polycubes_intro}, covers an abstract self-assembly model called \emph{polycubes}. Chapter~\ref{ch:polycubes1} details the model and the shapes we get when randomly sampling the input space. Chapter~\ref{ch:polycubes2} presents the results on the inverse problem; given a polycube shape, what input rules will assemble it and what is the least complex input you can find?

The second project takes a more detailed view of self-assembly, with Chapter~\ref{ch:oxview_intro} providing background on computer-aided design tools for nucleic acid structures, together with some related simulation models. Chapter~\ref{ch:oxview} then presents my contributions to \emph{oxView}, a web-based tool for the visualisation, design, and integration of DNA, RNA and protein structures.

Finally, Chapter~\ref{ch:conclusion} provides some concluding remarks, with a discussion on the results of both projects.

%Before moving on to the individual projects, however, the following sections will provide a general background on nucleic acid self-assembly and complexity.

%\section{Scope of the thesis}

