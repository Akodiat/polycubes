%\begin{savequote}[8cm]
%\textlatin{Cor animalium, fundamentum e\longs t vitæ, princeps omnium, Microco\longs mi Sol, a quo omnis vegetatio dependet, vigor omnis \& robur emanat.}

%The heart of animals is the foundation of their life, the sovereign of everything within them, the sun of their microcosm, that upon which all growth depends, from which all power proceeds.
  %\qauthor{--- William Harvey \cite{harvey_exercitatio_1628}}
%\end{savequote}

\chapter{The polycube codebase}
\label{ch:appendix_polycubes}

\minitoc

The code used for polycube assembly can be found at \url{https://github.com/akodiat/polycubes}. This code repository contains both the stochastic assembly code and the SAT solver code, as detailed in the sections below.

\section{Stochastic assembly code}
The polycube assembly model was implemented in both C++ and JavaScript. The C++ implementation enables the fast command-line evaluation of input rules required for the sampling performed in Chapter~\ref{ch:polycubes1}. Meanwhile, the JavaScript implementation makes it possible to assemble and visualise polycubes in a web browser, which was useful for both model validation and for public outreach activities.

\subsection{C++}
The stochastic assembly C++ code is found at 

\subsection{Python binding}

\begin{lstlisting}[language=bash]
  conda install -c conda-forge pybind11 eigen

  cd solve/py
  bash build_polycube_pybind.sh
\end{lstlisting}

\subsection{JavaScript}

\section{Polycube solver}

\subsection{Python}

\begin{lstlisting}[language=bash]
  python -m pip install python-sat[pblib,aiger]
\end{lstlisting}

In order to speedup sampling, the c++ binary can be called multiple times in parallel and merged with the merge python script. For the results covered in Chapter~\ref{ch:polycubes1}, the sampling was done on 100 concurrent nodes and merged.

The largest sampling done, the polyomino reference with 1e9 samples (described in Section~\ref{sec:refcalc}), took a total of 72 hours, 57 minutes and 58 seconds to sample with 100 copies each performing 1e7 samples in parallel. The merging script then took another 197 hours, 40 minutes and 24 seconds to run, but it clearly saved a significant amount of time compared to running a single sampling for 100 times longer (which would take almost 7300 hours, or 304 days).

Finally the analysis script took another 301 hours, 44 minutes and 17 seconds to run.

took 359 hours, 9 minutes and 44 seconds

\subsection{JavaScript}

\subsection{Patchy particle code}
\label{sec:appendix_patchy}
The modified version of oxDNA used to simulate torsional patchy particles can be found at \url{https://github.com/Akodiat/oxDNA_torsion}. To download and combile, run:

\begin{lstlisting}[language=bash]
  git clone https://github.com/Akodiat/oxDNA_torsion.git
  cd oxDNA_torsion
  mkdir build
  cd build
  cmake ..
  make -j4
  make romano
\end{lstlisting}

Simulation files for a given polycube shape can then be generated through the web console at \url{https://akodiat.github.io/polycubes}, by running:

\begin{lstlisting}[language=JavaScript]
  getPatchySimFiles(
    '070000070500868700000000', // Rule string
    1, // Number of assemblies
    'cube', // Name
    '/users/joakim/repo/oxDNA_torsion', // Path to oxDNA directory
    [.01,.02,.03,.04,.05,.06,.07,.08,.09,.1], // Temperature range
    0.1 // Density (used to generate configuration when the number
    // of assemblies is more than one)
  )
\end{lstlisting}


\section{Analysis}

\chapter{The oxView codebase}

The code for oxView can be found at \url{https://github.com/sulcgroup/oxdna-viewer}. It is written in TypeScript and compiled into JavaScript. The 3D visualisation is done with the help of the Three.js library, while the graphical user interface uses the Metro 4. 

To compile, you need both typescript and Node.js installed. Typing \verb|npm install| should install all required node modules.

Type \verb|tsc| to compile. 

\minitoc

